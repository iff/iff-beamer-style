\documentclass[table,10pt]{beamer}

\mode<presentation> {
  \usepackage{iff}
}

\usepackage[english]{babel}
\usepackage[latin1]{inputenc}
\usepackage{times}
\usepackage{listings}
\usepackage{colortbl}
\usepackage{verbatim}
\usepackage{amsmath}
\usepackage{amsfonts}
\usepackage{url}
\usepackage{fontspec}
\usepackage{xunicode} %Unicode extras!
\usepackage{xltxtra}  %Fixes

\usepackage{algpseudocode}
\algrenewcommand\algorithmicif{{\color{blue}\textbf{if}}}
\algrenewcommand\algorithmicelse{{\color{blue}\textbf{else}}}
\algrenewcommand\algorithmicthen{{\color{blue}\textbf{then}}}
\algrenewcommand\algorithmicdo{{\color{blue}\textbf{do}}}
\algrenewcommand\algorithmicforall{{\color{blue}\textbf{for all}}}
\algrenewcommand\algorithmicfor{{\color{blue}\textbf{for}}}
\algrenewcommand\algorithmicend{{\color{blue}\textbf{end}}}
\algrenewcommand\algorithmicprocedure{{\color{blue}\textbf{procedure}}}


%FIXME: need tikz to able to compile iff-style
\usepackage{tikz}
\usetikzlibrary{shapes,arrows,snakes,backgrounds}
\usetikzlibrary{mindmap,trees}
\usetikzlibrary{decorations.pathreplacing}
\usetikzlibrary{plotmarks}
\usetikzlibrary{calc}


% For Code listings
\lstnewenvironment{code}[1][]
{\textbf{Code Listing} \hspace{1cm} \hrulefill \lstset{language=C++,
basicstyle=\ttfamily\scriptsize, keywordstyle=\color{blue}\bfseries,
commentstyle=\color{mygreen}, stringstyle=\color{red}}}
{\hrule\smallskip}

\lstnewenvironment{smallcode}[1][]
{\lstset{language=C++, basicstyle=\ttfamily\scriptsize,
keywordstyle=\color{myblue}\bfseries,commentstyle=\color{mygreen},
stringstyle=\color{red}}}
{\smallskip}

\xdefinecolor{mygreen}{RGB}{0,220,0}
\xdefinecolor{myblue}{RGB}{26,150,255}


%%%%%%%%%%%%%%%%%%%%%%%%%%%%%%%%%%%%%%%%%%%%%%%%%%%%%%%%%%%%%%%%%%%%%%%%%%%%%%


\title[]{Iffs Beamer Style}

\author[Y. Ineichen]{Yves Ineichen}

\date{End of the world}



\begin{document}

\tikzstyle{na} = [baseline=-.5ex]

% For every picture that defines or uses external nodes, you'll have to
% apply the 'remember picture' style. To avoid some typing, we'll apply
% the style to all pictures.
\tikzstyle{every picture}+=[remember picture]

% By default all math in TikZ nodes are set in inline mode. Change this to
% displaystyle so that we don't get small fractions.
\everymath{\displaystyle}

% FIXME: set your code language
\lstset{language=C++, basicstyle=\small}



%%%%%%%%%%%%%%%%%%%%%%%%%%%%%%%%%%%%%%%%%%%%%%%%%%%%%%%%%%%%%%%%%%%%%%%%%%%%%%

\begin{frame} \titlepage \end{frame}

\begin{frame}

  \centering
  \begin{tikzpicture}
      \def\firstcircle  {(-1.0,0) circle (2.3cm)}
      \def\secondcircle {(2.5,0)  circle (2.3cm)}

      \begin{scope}
          \fill[red, fill opacity=0.3]    \firstcircle;
          \fill[blue, fill opacity=0.3]   \secondcircle;

          \draw node at (-1.0, 0.0) { \textbf{Trains} };
          \draw node at (2.5, 0.0) { \textbf{Speed of light} };

          \begin{scope}
            \clip \firstcircle;
            \fill<2->[acidgreen] \secondcircle;
          \end{scope}
      \end{scope}

  \end{tikzpicture}

  \pause
  \alert{Relativistic trains are awesome!}

  % and add a footnote with an appropriated citation
  \source{Relativistic Trains}{Sheldon Cooper, Phys.\ Rev.\ Stab., 2013.}

\end{frame}


% full page graphics with a title
\titledfullframegraphicH{\alert{xkcd}}{figures/slides}


% insert a special section frame
\sectionframe{Section 7}{lalalal}


\end{document}
